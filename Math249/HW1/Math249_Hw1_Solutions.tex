\documentclass{article}
\title{Math 249 - HW 1}
\author{Joshua Turcotti}
\usepackage{jt-shorthand}

\begin{document}
\maketitle
\section{Boxes in a Circle}


\begin{alphalist}
\item Consider $n$ boxes arranged in a circle, with $n$ boundary lines between them within the circle. Specifically, consider the boundary line $B\ind{0}$ at the very top, and the boundary line $B\ind{1}$ immediately clockwise of it. Let $C\ind{0}_n$ be the tilings of the $n$ boxes with monominoes and dominoes that place a boundary line at $B\ind{0}$ and $C\ind{1}_n$ be the tilings that place a boundary line at $B\ind{1}$. Since no domino or monomino can contain 2 consecutive boundary lines in its interior, $C_n = C\ind{0}_n \cup C\ind{1}_n$, and thus $\abs{C_n} = \abs{C\ind{0}_n} + \abs{C\ind{1}_n} - \abs{C\ind{0}_n \cap C\ind{1}_n}$. We note that the former two terms are simply $T_n$, the number of ways to tile $n$ consecutive boxes, and the latter term is $T_{n-1}$, the number of ways to tile $n-1$ consecutive boxes. Thus:
\begin{align*}
L_n &= T_n + T_n - T_{n-1}\\
&= F_{n+1} + F_{n+1} - F_n \\
&= F_{n+1} + F_{n-1}
\end{align*}
\item We recall the identity $F_{n+m} = F_{m+1}F_n + F_mF_{n-1}$, and compute (with initially arbitrary $k$ that will later be chosen to be $m-1$ so the desired identity is derived):
\begin{align*}
L_{m+n} &= F_{m+ n +1} + F_{m + n -1} \\
&= F_{k+1}F_{m + n + 1 -k} + F_kF_{m + n - k} + F_{k+1}F_{m + n - 1 -k} + F_kF_{m + n - 2 - k} \\
&= F_{k+1}L_{m + n - k} + F_kL_{m + n -1 - k} \\
&= F_{m}L_{n + 1} + F_{m-1}L_n
\end{align*}
\item \begin{align*}
F_{2n} &= F_nF_{n+1} + F_nF_{n-1} \\
&=F_n(F_{n+1} + F_{n-1}) \\
&= F_nL_n
\end{align*}
\end{alphalist}

\section{Permutations of a Multiset}

3 Methods for computing the permutations of a multiset with $n$ total elements in $m$ classes, with $n_i$ elements of class $1 \le i \le m$.
\begin{enumerate}
\item We may first choose the positions of the elements of class $1$, which may be done in $\binom{n}{n_1}$ ways, then the positions of class 2, which may be done in $\binom{n-n_1}{n_2}$ ways, similarly class 3 in $\binom{n-n_1-n_2}{n_3}$ ways, until we choose class $m$ in $\binom{n_m}{n_m}$ ways. Writing it out:
$$\frac{n!}{n_1!(n-n_1)!}\frac{(n-n_1)!}{n_2!(n-n_1-n_2)!}\ldots\frac{n_m!}{n_m!0!} = \frac{n!}{n_1!n_2!\ldots n_m!}$$
\item We may arrange all $n$ elements as if they are all unique, then note that we have overcountered by $n_i!$ for each class $1 \le i \le m$, so we take our total number of $n!$ permutations and divide by $n_i!$ for each class of redundancies, yielding the same expression as above.
\item We note that the identity holds when each class has but a single element (as all the denominator terms are 1 so it reduces to $n!$), so we perform vector induction by proving that the case of $(n_1, \ldots, n_m)$ follows from (WLOG) the case of $(n_1 - 1, n_2, \ldots, n_m)$, where $n_1 \ne 1$. We note that in the former case, we may choose the position of a single element of class 1 in $n$ ways, and arrange the rest of the elements in $\frac{(n-1)!}{(n_1-1)!n_2!\ldots n_m!}$ ways by the inductive hypothesis. This tells us that there are $\frac{n!}{(n_1-1)!n_2!\ldots n_m!}$ ways to permute the $(n_1, \ldots, n_m)$ multiset and choose a special element of class 1. To eliminate the choice of the special element of class 1, we group the $n_1$ different ways to choose that special element into a single case, and obtain the expression $\frac{n!}{n_1!\ldots n_m!}$ to count only the permutations of the multiset. 
\end{enumerate}

\section{Bijectivity of the Prufer Algorithm}

Let $T_X$ be the set of trees on the set $X \sub \N$ of nodes. Let $P_X$ be the prufer algorithm, that takes a tree in $T_X$ and outputs a sequence in $S_X := X^{\abs{X}-2}$. Let $R_X$ be the reverse Prufer algorithm, that takes a sequence $s \in S_X$ and returns a tree in $T_X$ with Prufer code $s$ (the existence of this algorithm was shown in lecture). We are thus given that $P_X \circ R_X = Id_{S_X}$, and we will show through induction on $\abs{X}$ that $R_X \circ P_X = Id_{T_X}$, establishing that $P_X$ is a bijection and $\abs{T_X} = \abs{S_X}$. First note that when $\abs{X} = 3$, there are exactly three trees (identified by the unique node with degree 2) and three prufer codes (each of which consists of a single node label). $P_X$ send a tree to the label of its degree-2 node, and $R_X$ sends a node label to a tree with that node as its unique degree-2 node, so the relations $P_X \circ R_X = Id_{S_X}$ and $R_x \circ P_X = Id_{T_X}$ are trivial. Assume the two relations hold for $\abs{X} = n$, and choose any new $X$ with $\abs{X} = n+1$. For the first identity, choose any $t =(X, E) \in T_X$. Where $a$ is the lowest labelled leaf, and $(a, b) \in E$, let $X' = X - \set{A}, E' = E - \set{(a, b)}, t' = (X', E')$. The definition of the prufer algorithm is that $P_X(t) = (b, P_{X'}(t'))$. Now $R_X(P_X(t)) = R_X(b, P_{X'}(t'))$. Since, $\abs{X'} = n$, $R_{X'}(P_{X'}(t')) = t'$. We use the property that the set of leaves in a tree is exactly the set of node labels that do not occur in its Prufer code, combined with the knowledge that the set of leaves in $t'$ is exactly the set of leaves in $t$ minus $a$ and (possibly) plus $b$, to conclude that $a$ is the least element of $X - \set{b} \cup P_{X'}(t')$. But $R_X(b, P_{X'}(t'))$ is computed exactly by taking the least element $a'$ of $X - \set{B} \cup P_{X'}(t')$, letting $X'' = X - \set{a'}$, and adding the node $a'$ and the edge $(a', b)$ to the tree $R_{X''}(P_{X'}(t'))$. We determined that $a$ is exactly this least element $a'$, so $R_X(P_X(t)) = R_X(b, P_{X'}(t'))$ is exactly the tree $R_{X'}(P_{X'}(t')) = t'$ (by the inductive hypothesis) with the node $a' = a$ and the edge $(a', b) = (a, b)$ added, which is exactly the tree $t$. Thus $R_X(P_X(t)) = t$, and since $t \in T_X$ was arbitrary and all possible (finite) $X$ are inductively covered, we may conclude that $R_X \circ P_X = Id_{T_X}$ for all $X$. As claimed at the beginning, we have shown that $P_X$ is a bijection from $T_X$ to $S_X$. 

\section{Counting Parking Functions}

\begin{alphalist}
\item We will show that the order in which the cars attempt to park does not affect the possibility of all cars parking. Since any permutation can be factoreds as a sequence of consecutive transpositions, it suffices to show that $a$ is a parking function iff $a'$ is a parking function where, for some $j$, $a'(j) = a(j+1), a'(j+1) = a(j)$, and $a'(i) = a(i)$ for all $i \ne j$. Assume that $a$ is a parking function, and let $k_1, k_2$ be the values of $a(j)$ and $a(j+1)$, choosen not respectively but such that $k_1 < k_2$. If some spot $i \ge k_1$, $i < k_2$ is available after the first $j-1$ cars have parked, then the same one of cars $j, j+1$ will park in that spot regardless of which tries first, so $a'$ is a parking function. If no such spot $i$ exists, then cars $j, j+1$ will occupy the first two available spaces at least $k_2$, and which one occupies which will switch but the overall viability will not be affected by which tried to park first. Thus $a'$ is a parking function. We have shown that transposing the parking order of two cars in a parking function yields another parking function, and it cannot be the case that transposing a non-parking function results in a parking function because then we could simply perform the same transposition again and conclude that the original non-parking function is indeed a parking function, so we can conclude that $a$ is a parking function iff $a'$ is a parking function whenever $a$ and $a'$ are transpositions. With this in mind, we constrain our attention to the case in which cars park in ascending order of preference. If it is the case that $b(i) \le i$ for every $i$, then it is easy to see that car $i$ will park in spot $i$ and so clearly $b$ is a parking function. If $b(j) > j$ for some $j$, then all of the cars $j, j+1, \ldots, n$ will attempt to park in the interval $b(j), \ldots, n$, which cannot happen and thus $b$ is not a parking function. This allows us to conclude that any $a$ is a parking function iff its increasing arrangement $b$ is a parking function, which occurs iff $b(i) \le i$ for all $i$.
\item Consider the circular parking case, in which each of $n$ cars chooses one of the $n+1$ spots to park and works clockwise to find the first available. If no car parks in space $n+1$, then no car desired to park in space $n+1$ and no car that desired to park in a space less than $n+1$ was forced to look past the first $n$. If a car parks in space $n+1$ then it either desired space $n+1$ or desired a space at most $n$ but every space at least its desire and at most $n$ was taken. Thus a function $[n] \to [n+1]$ is a regular, $[n] \to [n]$ parking function iff is a circular parking function that results in no car parking in space $n+1$. By symmetry, the number of functions $[n] \to [n+1]$ that result in no car parking in space $n+1$ is exactly the same as the the number of functions that result in no car parking in space $i$ for any $i$, and every function $[n] \to [n+1]$ results in exactly one empty space, so exactly $\frac{1}{n+1}$ of all functions $[n] \to [n+1]$ result in no car parking in space $n+1$, and thus exactly $\frac{1}{n+1}$ of all functions $[n] \to [n+1]$ are regular parking functions. This gives us the desired count of $(n+1)^{n-1}$ parking functions $[n] \to [n]$. 
\end{alphalist}

\section{Other Catalan Countings}

We know from lecture that the Catalan numbers $C_n$ count the number of length $2n$ Dyck paths: (N, E)-only lattice paths from $(0, 0)$ to $(n, n)$ that do not dip below the line $x=y$. We will show that they also count two similar sets.
\begin{alphalist}
\item Establish a bijection between words $w \in \set{0, 1}^{2n}$ with $n$ 0s and $n$ 1s such that any prefix contains at least as many 1s as 0s and length $2n$ Dyck paths by sending 1s to N moves and 0s to E moves. If no prefix of $w$ contains more 0s than 1s and $w$ itself contains an equal number of 1s and 0s, then the corresponding path will never dip below $x=y$ and will terminate at $(n, n)$. Similarly, any Dyck's path corresponds to a words with equal 1s and 0s and no prefix with more 0s than 1s. Thus the Catalan numbers $C_n$ also count such words.
\item Map a sequence $a_1, \ldots, a_n$ to a path from $(1, 1)$ to $(n+1, n+1)$ that contains only vertical and East steps, and contains exactly the $n$ East steps from $(i, a_i)$ to $(i+1, a_i)$ for each $i$, filled in with $n$ vertical steps to link them. Up to mirroring over the line $x=y$ these are exactly the set of length $2n$ Dyck's paths (shifted by +1, +1), because $a_i \le i$ gaurantees no North steps are taken above $x=y$ and $a_i \le a_{i+1}$ gaurantees no South steps are needed. A Dyck's path can be converted to a sequence $a_1, \ldots, a_n$ by shifting it (+1, +1), and letting $a_i$ be the maximum $j$ such that the path contains $(i, j)$, giving us the property that $a_i \le i$ and that $a_i \le a_{i+1}$. Thus the sequences $a_1, \ldots, a_n$ with these two properties are counted by the Catalan numbers $C_n$.
\end{alphalist}

\section{Divisor-Free Partitions}

Given $n \ge 1$, $m \ge 2$, we will show that the partitions $P_d$, of $n$ into positive integers repeated fewer than $m$ times each, and $P_f$, of $n$ into arbitrarily repeated positive integers that are free of the divisor $m$, are equinumerous. Let $g$ be the map sending $\g \in P_d$ to a partition in $P_f$ constructed by replacing each part $p = qm^r$, expressed such that $m \nmid q$, with $m^r$ copies of $q$. Let $f$ be the map sending $\g \in P_f$ to a partition in $P_d$ constructed with $c_r$ copies of $qm^r$ for each $q$ that occurs as a part of $\g$ exactly $c = (c_0 + c_1m + c_2m^2 + \ldots)$ times, expressed such that $c_r < m$. It suffices to show that $f \circ g = Id_{P_d}$ and $g \circ f = Id_{P_f}$. For the former identity, choose $\g \in P_d$. For each $q$ free of the divisor $m$, let $c\ind{q}_r$ be the number of times $qm^r$ occurs as a part of $P_d$, which is clearly less than $m$. $g(\g)$ will contain $c\ind{q} = \sum_r c\ind{q}_rm^r$ copies of $q$, and since there is a unique way to express $c\ind{q}$ as such a sum of powers of $m^i$, $f(g(\g))$ will contain $c\ind{q}_r$ copies of $qm^r$ for each $r$. Since this enumeration over $m$-free $q$ partitions all possible parts of $\g$, we have shown that $f(g(\g)) = \g$. For the latter identity, choose $\g \in P_f$. For each $q$ that occurs as a part of $\g$, let $c\ind{q}$ be the number of times it occurs. $f(\g)$ will express $c\ind{q} = \sum_r c\ind{q}_rm^r$ and contain $c\ind{q}$ copies of $qm^r$ for each $r$. $g(f(\g))$ will thus contain $c\ind{q}$ copies of $q$, and thus be identical to $\g$. We have shown both desired identities, and thus $g$ is a bijection so the two sets of partitions are indeed equinumerous.

\section{Counting Derangements}

Consider the derangements of $[n]$, and let $D(n)$ be their number. Combinatorially, we observe that the element 1 can either be part of a cycle of length 2, or of length greater than 2. There are $(n-1)D(n-2)$ ways to derange $[n]$ placing 1 in a cycle of length 2, as we must choose the other element of 1's cycle and an independent derangement of the remaining $n-2$ elements. There are $(n-1)D(n-1)$ ways to derange $[n]$ placing 1 in a cycle of length greater than 2, as we can choose any derangement of the remaining elements, then choose 1 element to precede the element 1 after its insertion. This is a bijection as distinct derangements of the $n-1$ elements or distinct choice of which of the $n-1$ elements to precede 1 will yield a necessarily different permutation. In total, we have counted that $D(n) = (n-1)(D(n-1) + D(n-2))$. We can also procede with the principle of inclusion exclusion. We begin with the $n!$ permutations of $[n]$. We then remove the $(n-1)!$ permutations that fix $i$ for each $i \in [n]$, removing $n(n-1)!$ permutations total. We then replace the $(n-2)!$ permutations of $[n]$ that fix $i, j$ for each of the $\binom{n}{2}$ choices of distinct $i, j$. We exhaustively proceed to obtain:
\begin{align*}
D(n) &= \sum_i\qty((-1)^i(n-i)!\binom{n}{i}) \\
&= n!\sum_{i \le n}\frac{(-1)^i}{i!} \\
&= n\qty((n-1)!\sum_{i \le n-1}\frac{(-1)^i}{i!}) + (-1)^n \\
&= nD(n-1) + (-1)^n \\
&= (n-1)D(n-1) + D(n-1) + (-1)^n \\
&= (n-1)D(n-1) + (n-1)D(n-2) + (-1)^{n-1} + (-1)^n \\
&= (n-1)(D(n-1) + D(n-2))
\end{align*}
\section{Fresh Faces}

Given $k$ out of the $n$ couples, there are exactly $(2n - k)!2^k$ ways to seat the $2n$ people such that all $k$ of the chosen couples sit next to each other. This is computed by treating each couple as a unit instead of 2 individual people then counting permutations, then counting the order in which each couple sits. Now, using inclusion-exclusion, the total number of ways to seat the $2n$ people such that no couples sit next to each other is:
$$\sum_{k = 0}^n\binom{n}{k}(2n -k)!(-2)^k$$

\section{Permutation Signing}

First we will demosntrate the inductive approach. Letting $c(n, k) = 0$ when $k = 0$ or $k > n$, the identity $c(n, k) = c(n-1, k-1) + (n-1)c(n-1, k)$ can be seen to hold in all cases by observing that of the $k$-cycle permutations of $n$ elements, $c(n-1, k-1)$ of them place element $n$ in a 1-cycle, and all the rest can be obtained by choosing an arbitrary $k$-cycle permutation of the elements $1, \ldots, n-1$ and then choosing one of those $n-1$ to precede $n$ upon its insertion into one of the $k$ cycles. Assume that the given identity holds for $n-1$, and we will demonstrate the case for $n$:
\begin{align*}
\sum_{k=1}^n(-1)^{n-k}c(n, k) &= \sum_{k=1}^n(-1)^{n-k}\qty(c(n-1, k-1) + (n-1)c(n-1, k)) \\
&= \sum_{k=1}^{n-1}(-1)^{n-k-1}c(n-1, k) - (n-1)\sum_{k=1}^{n-1}(-1)^{n-k-1}c(n-1, k) \\
&= 0 - (n-1)0\\ 
&= 0
\end{align*}
Now, we can define the sign of a permutation of $[n]$ as $-1$ raised to the number of cycles. We define the involution $\i: \s \mapsto \s\circ(1~2)$, which composes any permutation with the transposition $(1~2)$. $\i^2(\s) = \s\circ(1~2)\circ(1~2) = \s$, so it is indeed an involution, and further it sends any $\s$ in which 1 and 2 are in separate cycles to one in which they share a cycle, and vice versa, while leaving all other cycles untouched. Thus it is a sign-reversing involution that additionally has no fixed points, as no permutation is equal to itself composed with $(1~2)$. Since the sum in the identity is exactly $\sum_\s \text{sgn}~\s$, we can conclude it equals 0.

\section{Adjacency-Free Partitions}

First, we note that for any choice of $k$ pairs of the form $i, i+1 \in [n]$, there are $B(n-k)$ ways to partition $[n]$ such that in each of the pairs, both elements are in the same block. It may seem as if some distinction should be made between overlapping and non-overlapping sets of pairs - but in fact it makes no difference - as we can consider each element of $[n]$ to initially be its own "atom", and then every pair that we consider merges two atoms, decreasing the total number by 1 and leaving us with $n-k$ atoms to be partitioned into blocks of atoms, or equivalently blocks of elements that unite each pair. With this identity ($B(n-k)$ ways to partition $[n]$ when $k$ pairs are chosen to be forcibly united) justified, we can use inclusion-exclusion to write down the total number of partitions of $[n]$ such that no pair of the form $i, i+1$ is united:
$$A(n) = \sum_{k = 0}^{n-1}(-1)^k\binom{n-1}{k}B(n-k)$$
\end{document}

