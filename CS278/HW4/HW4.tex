\documentclass{article}

\usepackage{jt-shorthand}

\title{CS 278 - Homework 3}
\author{Joshua Turcotti}
\begin{document}
\maketitle

\newcommand{\bset}{\set{0, 1}}
\section{SAT Algorithms for $ACC^0$}
\begin{alphalist}
\item For any set $S \sub [n]$, let $a_S$ be corresponding coefficient in $P$. For any set $S \sub [n]$ let $a\ind{0}_S$ and $a\ind{1}_S$ be the corresponding coefficients in $P_0$ and $P_1$, respectively. Now:
  \begin{align*}
    a\ind{0}_S &= a_S \\
    a\ind{1}_S &= a_{S \cup \set{n}}
  \end{align*}
\item For all $x \in \bset^n$ with $x_n = 0$, $P(x) = P_0(x)$, so no computation is needed, just a copy, time $O(1)$. For all $x \in \bset^n$ with $x_n = 1$, $P(x) = P_0(x) + P_1(x)$. We assume to have been given all of the latter such values, so only an $O(1)$ addition must be performed for each of the $2^{n-1}$ such $x$ with $x_n = 1$. Thus for all $x \in \bset^n$, an $O(1)$ operation suffices to compute $P(x)$ given all values of $P_0$ and $P_1$, yielding $O(2^n)$ overall time to compute all values of $P$ over $x \in \bset^n$.
\item Since each round is accumulated time $O(2^n)$ across the branches, and the entire computation is $n$ rounds deep, the algorithm takes time $O(2^n n)$ (and can be done in-place, i.e. $O(1)$ space complexity).
\item Once we have a list of all $2^n$ values that $P$ takes on $x \in \bset^n$, we simply check $\Psi$ on each of those values, and iff we encounter some $1 \le w_0 \le w$ that occurs as in the range of $P$ such that $\Psi(w_0) = 1$, we conclude that $C'$ is satisfiable. 
\end{alphalist}
\section{SAT Algorithms - Faster than $2^n$}
\begin{alphalist}
\item  We can phrase satisfiability for $C$ as: $$\exists x \in \bset^n: C(x) = 1$$ Equivalently: $$\exists x \in \bset^{n-t}: \exists z \in \bset^t : C(z, x) = 1$$ Equivalently: $$\exists x \in \bset^{n-t}: \bigvee_{z \in \bset^t} C(z, x) = 1$$ Equivalently: $$\exists x \in \bset^{n-t}: C^*(x) = 1$$ But this last predicate is exactly satisfiability for $C^*$, so we are done.
\item The circuit size of $C^*$ is $O(2^t s)$; bounding by one copy of $C$ per $z \in \bset^t$. It depends on $n-t$ input bits, and has depth $d+1 = O(1)$.
\item To apply the SAT algorithm from part 1, we need to show that the circuit size, $s' = O(2^t s)$ is $poly(n-t)$. Since $t = O(\log n)$, $poly(n-t) = poly(n)$. Now $s = poly(n)$, which is given, and $2^t = poly(n)$, which follows from the definition of $t = 101\log(n)$, are sufficent to conclude that $s'$ is $poly(n-t)$, so we may apply the SAT algorithm and conclude that satisfiability for $C^*$ can be determined in time $O(2^{n-t}(n-t)) = O(\frac{2^n}{n^{101}}(n-t)) = O(\frac{2^n}{n^{100}})$.
\end{alphalist}
\section{Paralllel Repetition of $AM$}
We will show the lifting of perfect completeness and strengthened error for soundness to the parallel version of $AM$. For completeness, assume $x \in L$, and note that if Merlin uses $k$ independent copies of $\Pi$ to respond to the $k$ messages $r_i$ received, then Arthur will always compute $A(x, y_i, r_i) = 1$ for all $i$ by the perfect completeness of $\Pi$. For exponentially strengthened soundness, let $x \not\in L$, and assume that Merlin manages to respond with a scheme that is independent between messages (we assume cheating-free), yet Arthur accepts with probability strictly greater than $(1/3)^k$. By independence, the probability over $r$ that Arthurt accepts is equal to the product of the probabilities over $r$ that $A(x, y_i, r_i) = 1$ for each each $i$. But now for some $i$, this probability must be greater than $1/3$, which contradicts the assumptions of $\Pi$. We can conclude that for any scheme under which Merlin answers, the probability that Arthurt accepts is at most $(1/3)^k$. 

\end{document}