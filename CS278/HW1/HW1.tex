\documentclass{article}
\title{CS 278 - HW1}
\author{Joshua Turcotti}
\usepackage{jt-shorthand}
\begin{document}
\maketitle

\section{Space Hierarchy Theorem}

\renewcommand{\L}{\mathbf{L}}
\newcommand{\NL}{\mathbf{NL}}
\newcommand{\coNL}{\mathbf{coNL}}
\newcommand{\NSP}{\mathbf{NSPACE}}
\newcommand{\coNSP}{\mathbf{coNSPACE}}
\newcommand{\U}{\mathcal{U}}
\newcommand{\SP}{\mathbf{SPACE}}

Assume $f(n) = O(g(n))$. Let $h(n) = \sqrt{f(n)g(n)}$ such that $f = o(h)$ and $h = o(g)$. Let $\U$ be the universal turing machine given in the problem statment. Let $L$ be exactly the language consisting of those $x \in \set{0, 1}^*$ such that $\U$ can run the TM $M_x$ on input $x$ and terminate with $h(n)$ space and output 0. We claim that $L \in \SP[g] - \SP[f]$. Since $h = O(g)$, $L \in \SP[g]$ is trivial. For the sake of contradiction, assume $L \in \SP[f]$. Now there exists some TM $M$ that can determine $L$ in space $f$. Further, we know that $\U$ can simulate $M$ with space $Cs$ on all inputs $x$ for which $D$ requires space $s$. Choose $n_0$ such that $n > n_0$ implies $h(n) > Cf(n)$. Now choose $x$ of size greater than $n_0$ such that $M = M_x$. It must be the case that $M$ runs and terminates on input $x$ with space $f(x)$, so $\U$ can simulate the execution of $M$ on $x$ with space $Cf(x) < h(x)$, which implies that $L$ contains $x$ if and only if the language determined by $M$ does not. But this is a contradiction, as $M$ was defined to determine exactly the language $L$. Thus $L \not \in \SP[f]$, and so our assertion $L \in \SP[g] - \SP[f]$ holds.

\section{Log-Space Reductions}

In a lemma that will be relevant to both parts a and b below, we note that any $O(\log(\abs{x}))$-space deterministic turing machine with a single read-only input tape, the possible presence of a write-once output tape, and an arbitrary positive (but constant) number of read-write work tapes can be simulated by a $O(\log(\abs{x}))$-space deterministic turing machine with a single read-write work tape. If the former machine machine has $n$ read-write work tapes, then we can use the $(kn+l)$th cell in the single read-write work tape of the latter machine (for $l < n$) to simulate the $k$th cell of the $l$th tape of the former machine. If the former machine was bound to have $O(f(\abs{x}))$ cells used per tape, the latter machine will still be bound to have $O(nf(\abs{x})) = O(f(\abs{x}))$ cells used per tape. Now we proceed.

\begin{alphalist}
\item We consider the Turing machine $M$ that computes $f(x)$ from $x$ using $O(\log\abs{X})$ space, writing $f(x)$ to a write-once output tape. We will construct $M'$ that computes the $i$th bit of $f(x)$ using $O(\log\abs{X})$ space, using no output tape. Specifically, we note from above that we may give $M$ a second read-write work tape as long as no more than $O(\log\abs{X})$ of its cells are used. This tape is initially set entirely to 0. $M'$ simulates $M$ exactly, except that whenever $M$ would write a bit to the output tape, instead $M'$ checks the current value of the entire second work tape and if it is equal to $i$, outputs that bit, and otherwise performs the binary addition algorithm on that entire tape to add 1 to its value. Since $i$ is bounded above by $ \abs{x}^c$, this tape will never have more than $c\log\abs{x}$ bits written to it, establishing that $M'$ indeed only uses $O(\log\abs{X})$ space.
\item We consider the Turing Machine $M$ that rejects or accepts inputs $x$ from a read-only input tape, using a read-write work tape with $O(\log\abs{x})$ cells, to determine the language $B \in \L$. Given a log-space computable reduction $f$ from $A$ to $B$, we will show $A \in \L$ by constructing a Turing Machine $M'$ to deteremine $A$. Beginning with $M$, add two more read-write work tapes, $t_1$ and $t_2$, both initialized to 0. To run $M'$ on the input $x$, simulate $M$, except that whenever $M$ moves its input head to the right (resp. left), increment (resp. decrement) the binary number represented on $t_1$, and whenever $M$ reads a bit from the input tape, run the procedure from part a on the input pair ($x$, contents of $t_1$) using the work tape $t_2$, and then procede as if the output from that procedure was the bit read from the input. In this way, $M'$ will act exactly as $M$ on the remaining work tape, and on its state when not subroutining the procedure for part a, and thus will accurately determine if $f(x) \in B$, and thus if $x \in A$, using $O(\log\abs{x})$ space. This establishes that $A \in \L$.
\end{alphalist}

\section{Immerman Szelepcsenyi's Theorem}

\begin{alphalist}
\item To show that $A' \in \NL$, we must show that there exists a Turing Machine $M'$ such that $y \in A' \iff \exists z: M'(y, z) = 1$ and that runs with $O(\log\abs{y})$-space. Since we know $A \in \NSP[n]$, let $M$ be the machine such that $x \in A \iff \exists z: M(x, z) = 1$ and that runs with $O(\abs{x})$ space. Let $M'$ accept the pair $(y, z)$ iff $M$ accepts the first $\log(\abs{y})$ bits of $y$ and all the remaining bits of $y$ are 0. The first check takes space $O(\log\abs{y})$ and the second check takes space $O(1)$, so $M'$ is an $O(\log\abs{y})$-space machine that (by the definition of $A'$) determines $A'$, proving $A' \in \NL$. 
\item Since $\NL = \coNL$, we already have $A' \in \coNL$. This gives us the existence of a Turing Machine $N'$ such that $y \in A' \iff \forall z: N'(y, z) = 1$ that runs with $O(\log\abs{y})$ space. Define the Turing Machine $N$ that accepts $x$ iff $N'$ accepts $(x, 0^{2^{\abs{x}} - \abs{x}})$. 
We note that by the definition of $A'$, $N$ determines exactly the language $A$, and runs with space $O(\log{2^{\abs{x}}}) = O(\abs{x})$, proving $A \in \coNSP[n]$. Since $A$ was an arbitrary language in $\NSP[n]$, we have shown $\NSP[n] \sub \coNSP[n]$
\item Let $A \in \coNSP[n]$. Then $\bar{A} \in \NSP[n]$, so by part b $\bar{A} \in \coNSP[n]$. But then by the definition of $\coNSP[n]$ as the set of complements of $\NSP[n]$, $\bar{\bar{A}} = A \in \NSP[n]$. This allows us to conclude $\coNSP[n] \sub \NSP[n]$, and thus $\coNSP[n] = \NSP[n]$.
\end{alphalist}

\end{document}

